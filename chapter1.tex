\chapter{Introduction}
\pagenumbering{arabic}\hspace{3mm}

\section{Cloud}

Cloud computing is the on-demand availability of computer
system resources, especially data storage and computing power,
without direct active management by the user.
The term is generally used to describe data centers available
to many users over the Internet.

Cloud services refer to any IT services that are provisioned 
and accessed from a cloud computing provider. 
This is a broad term that incorporates all delivery and 
service models of cloud computing and related solutions. 
Cloud services are delivered over the internet and accessible globally from the internet. There are three basic types of cloud services:
\begin{itemize}
    \setlength\itemsep{-1em}
    \item Software as a Service (SaaS)
    \item Platform as a service (PaaS)
    \item Infrastructure as a service (IaaS)
\end{itemize}

\section{Cloud Services}

\subsection{SaaS}
SaaS is a software distribution model in which applications are hosted by a vendor or service provider and made available to customers over a network, typically the internet. Examples include G Suite -- formerly Google Apps, Microsoft Office 365, Salesforce and Workday.

\subsection{PaaS}
PaaS refers to the delivery of operating systems and associated services over the internet without downloads or installation. The approach lets customers create and deploy applications without having to invest in the underlying infrastructure. Examples include Amazon Web Services' Elastic Beanstalk, Microsoft Azure -- which refers to its PaaS offering as Cloud Services -- and Salesforce's App Cloud.

\subsection{IaaS}
IaaS involves outsourcing the equipment used to support operations, including storage, hardware, servers and networking components, all of which are made accessible over a network. Examples include Amazon Web Services, IBM Bluemix and Microsoft Azure.

\section{Experimental Cloud using Commodity Hardware}

The objective of this project is to create an experimental
cloud by repurposing commodity hardware. The cloud we create would
be made available to students as virtual desktops which may be used
to host web services which can vary from simple static page to
complex web applications. 

\section{Organization of The Report}

This chapter provides an overview of cloud computing and cloud services.
In the next chapter we will introduce MaaS(Metal as a Service), which is a relatively new approach for cloud based service. 
In chapter 3, we will discuss some of the tools that we need to be familiar with to break the ice. In chapter 4, we will discuss the approach by which we can create a MaaS based cloud environment. And finally in chapter 5, we conclude with some future works.
